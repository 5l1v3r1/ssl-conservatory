% iSEC Whitepaper template
\documentclass{article}
\usepackage{isec_whitepaper} % iSEC formatting
\setlength{\footnotesep}{0.5cm}
\setlength{\skip\footins}{1cm}

% Just edit title and author
\title{\scshape Everything You've Always Wanted to Know About Certificate 
    Validation With OpenSSL\\ (but Were Afraid to Ask)}
\author{\em Alban Diquet --- alban[at]isecpartners[dot]com\\
\\
iSEC Partners, Inc\\
123 Mission Street, Suite 1020\\
San Francisco, CA 94105\\
\url{https://www.isecpartners.com}}
\date{\today}
\makeindex

% End of preamble
\begin{document}
\maketitle
\thispagestyle{fancy}
\begin{abstract}

Applications that need to securely communicate with a remote server commonly
rely on the TLS\footnotemark protocol. Various libraries are available for 
developers who want to leverage this protocol --- one of the most popular 
being OpenSSL. Securely connecting to a server is not merely a matter of 
simply ``enabling'' TLS, but requires a great deal of care when implementing 
the client application in order to actually get the security benefits of using 
TLS. Most critically, TLS clients must validate the server's identity by 
verifying its X.509 certificate upon connection. Certificate validation is an 
extremely complex task, even when relying on a library such as OpenSSL. This 
paper provides application developers with clear and simple guidelines on how 
to perform certificate validation within a TLS client using OpenSSL.

\footnotetext{Note that although OpenSSL supports both SSL and TLS, we refer 
    almost exclusively to TLS in this document, as ``SSL'' is no longer 
    current and should not be used.} 

\end{abstract}

%--------------------------------------------------
\section{Introduction}

Although TLS is widely known for encrypting communication between two
parties, verifying the identify of the server by validating its X.509
certificate is critical. Without proper certificate validation, the TLS client
cannot ensure that it is talking to the real server instead of an attacker 
masquerading as the server. Failing to perform this step renders meaningless 
the confidentiality and integrity guarantees of TLS\footnotemark.

\footnotetext{It is sometimes assumed that ``something is better than 
    nothing'' with SSL/TLS, because it obscures traffic from a passive network 
    monitor; however, given the negligible technical and cost differences 
    between implementing passive versus active network interception, any 
    attacker capable of one is almost certainly capable of both. All network 
    attackers should be assumed to be ``active''.}

Implementing certificate validation within a TLS client application is
critical, but it's also a complex task. At a high-level, this validation is
a two-step process:

\begin{enumerate}
    \item The server's certificate chain must be validated.
    \item The server's hostname must be verified.
\end{enumerate}

Numerous applications\footnotemark relying on OpenSSL failed to properly
implement one or both of these steps, resulting in the application's TLS
connections being vulnerable to man-in-the-middle attacks, thereby defeating
the purpose of using TLS.

\footnotetext{One notable example is Python: \url{http://bugs.python.org/issue1589}} 


Application developers are not the only ones to blame. OpenSSL's APIs are
sometimes poorly documented or confusing, and there is very little information
available that describes how to properly do the validation. Additionally, as
opposed to other TLS libraries, OpenSSL does not provide any utility function
to perform the second step, validating the server's hostname\footnote{Helper
functions to perform hostname validation were added to OpenSSL 1.1.0 but
this version has not been released yet.}. Requiring application developers to
implement hostname validation themselves has led to numerous critical security
vulnerabilities.

Ideally, developers should instead rely on a well-tested, higher level library
such as libCURL\footnote{\url{https://curl.haxx.se/libcurl/}} that already
performs certificate validation. Otherwise, if OpenSSL must be used directly,
\textbf{\ref{sec:certchain}} and \textbf{\ref{sec:hostname}}
of this paper provide detailed guidelines and sample code on how to perform
this validation.

Rather than handling all the possible scenarios or corner cases that could
arise when validating X.509 certificates, the goal is to provide simple and
secure guidelines that should work for most applications. Known limitations
are described in \textbf{\ref{sec:limitations}}.

The code snippets provided in this paper implement a simple HTTPS client that
securely connects to \url{https://www.google.com}. The full working client as
well as the code snippets provided are all available on iSEC Partners' GitHub
at \url{https://github.com/iSECPartners/ssl-conservatory/}.


%--------------------------------------------------
\section{Validating the certificate chain}
\label{sec:certchain}
When connecting to a server, a TLS client first must validate the certificate
chain returned by the server in order to ensure that the server's certificate
is trusted.


%--------------------------------------------------
\subsection{Use OpenSSL's default validator}
\label{sec:defaultvalidator}

Validating the server's certificate chain with OpenSSL is relatively
straight-forward. OpenSSL provides a default built-in validator that, given a
list of trusted CA certificates, verifies the validity of a certificate chain
by performing numerous checks including ensuring that:

\begin{itemize}
    \item The server certificate chains up to a trusted CA certificate.
    \item Certificates within the chain have valid signatures.
    \item Certificates within the chain are not expired.
\end{itemize}

It is possible to override this default validation using the 
{\tt SSL\_CTX\_set\_cert\_verify\_callback()}\footnote{\url{https://www.openssl.org/docs/ssl/SSL_CTX_set_cert_verify_callback.html}}
function. However, this should \textbf{never} be attempted, because validating
a certificate chain is an extremely difficult task; OpenSSL's built-in
validator should always be used.

The rest of this section will focus on how to properly configure and use the
OpenSSL default validator.

%--------------------------------------------------
\subsection{Configuring OpenSSL's default certificate chain validator}

\subsubsection{Enabling certificate chain validation}

First, peer certificate validation has to be enabled within the 
SSL\_CTX\footnote{\url{https://www.openssl.org/docs/ssl/SSL_CTX_set_verify.html}} 
object:

\begin{lstlisting}[style=code,language=C,numbers=none,caption={}]
void SSL_CTX_set_verify(SSL_CTX *ctx, int mode, int (*verify_callback)(int, X509_STORE_CTX *));
\end{lstlisting}

\begin{itemize}

    \item The {\tt mode} should be set to {\tt SSL\_VERIFY\_PEER} to enable certificate 
        validation.

    \item An optional {\tt verify\_callback} can be provided, which will be called
        after OpenSSL's default validator has checked a certificate within the
        certificate chain. It can be used to get more information about a certificate
        error in case validation failed, or to do additional checks on the
        certificates. It should be noted that this is a different callback than the
        one described in \vref{sec:defaultvalidator}.

\end{itemize}


\subsubsection{Configuring the trust store to be used}

A file or a folder containing a list of PEM-encoded trusted CA certificates
must be provided to the SSL\_CTX\footnote{\url{https://www.openssl.org/docs/ssl/SSL_CTX_load_verify_locations.html}}:

\begin{lstlisting}[style=code,language=C,numbers=none,caption={}]
int SSL_CTX_load_verify_locations(SSL_CTX *ctx, const char *CAfile, const char *CApath);
\end{lstlisting}

The appropriate trust store to use is dependent upon the type of application:

\begin{itemize}

    \item Most applications only have to connect to one or a
        few application servers. Therefore, the trust store should
        only contain the CA certificates needed to connect to those
        servers. Restricting the list of trusted CA certificate
        in such way is a security practice called certificate
        pinning.\footnote{\url{http://blog.thoughtcrime.org/authenticity-is-broken-in-ssl-but-your-app-ha}}

    \item Applications that need to be able to connect to any server
        on the Internet (such as browsers) could instead rely on Mozilla's
        list of root certificates used in Firefox. The list is
        not directly available in PEM format, but Adam Langley has written a
        tool\footnote{\url{https://github.com/agl/extract-nss-root-certs}}
        to extract and encode the certificates so that they can directly
        be used with OpenSSL. Using such a large trust store in an
        application is far from ideal, because it then requires the
        application developers to keep the trust store up to date.
        For example, in the event of a certificate authority being
        compromised,\footnote{\url{https://googleonlinesecurity.blogspot.com/2011/08/update-on-attempted-man-in-middle.html}}
        application users will be exposed until the application's trust store is
        patched.

\end{itemize}

\subsubsection{Sample Code}

The following sample code enables certificate validation within our test
client. The trust store we're going to use is a file called 
{\em VerisignClass3PublicPrimaryCertificationAuthority.pem} which contains the
Verisign CA certificate used to sign www.google.com's server certificate.

This PEM file was obtained by browsing to \url{https://www.google.com} using
Firefox, inspecting the certificate chain, and exporting the CA certificate.

\begin{lstlisting}[style=code,language=C,numbers=none,caption={}]
#define TRUSTED_CA_PATHNAME "VerisignClass3PublicPrimaryCertificationAuthority.pem"

ssl_ctx = SSL_CTX_new(TLSv1_client_method());

// Enable certificate validation 
SSL_CTX_set_verify(ssl_ctx, SSL_VERIFY_PEER, NULL);

// Configure the trust store to be used
if (SSL_CTX_load_verify_locations(ssl_ctx, TRUSTED_CA_PATHNAME, NULL) != 1){
	fprintf(stderr, "Couldn't load certificate trust store.\n");
	return -1;
}
\end{lstlisting}


%--------------------------------------------------
\subsection{Checking the result of the certificate chain validation}

\subsubsection{Obtaining the result of the validation}

After enabling certificate chain validation and providing a trust store within
the SSL\_CTX object, subsequent SSL objects generated using this SSL\_CTX will
automatically perform certificate chain validation. More specifically, the SSL
handshake will fail if the certificate chain sent back by the server does not
pass the validation.

If the handshake fails, the following 
function\footnote{\url{http://www.openssl.org/docs/ssl/SSL_get_verify_result.html}}
can be used to recover the result of the certificate chain validation:

\begin{lstlisting}[style=code,language=C,numbers=none,caption={}]
long SSL_get_verify_result(const SSL *ssl);
\end{lstlisting}

There are numerous possible return values all described in the OpenSSL 
documentation\footnote{\url{https://www.openssl.org/docs/apps/verify.html}}.
Notable values include:

\begin{itemize}

    \item {\tt X509\_V\_OK}: The verification succeeded or no peer
        certificate was presented. If the handshake failed, receiving this
        return value means that the handshake did not fail because of a
        certificate error. Other reasons include the server rejecting the
        handshake, network congestion, etc...

    \item {\tt X509\_V\_ERR\_DEPTH\_ZERO\_SELF\_SIGNED\_CERT}:
        Verification failed because the server's certificate is self-signed and
        isn't part of the trust store.

    \item {\tt X509\_V\_ERR\_SELF\_SIGNED\_CERT\_IN\_CHAIN}:
        Verification failed because the CA certificate up the certificate chain
        is not part of the trust store.

\end{itemize}


\subsubsection{Recovering the server's certificate} 

If the handshake succeeded, the application must 
retrieve\footnote{\url{http://www.openssl.org/docs/ssl/SSL_get_peer_certificate.html}}
the server certificate:

\begin{lstlisting}[style=code,language=C,numbers=none,caption={}]
X509 *SSL_get_peer_certificate(const SSL *ssl);
\end{lstlisting}

This ensures that a certificate was actually sent back by the server. If the
handshake succeeded but no certificate was received, it means that the SSL
connection is using an insecure anonymous cipher suite for transport security.
Such cipher suites are vulnerable to man-in-the-middle attacks and should never 
be used, as it provides no server authentication.

\subsubsection{Sample Code}

The following sample code performs the TLS handshake. If the handshake fails,
it then checks whether the failure was due to a certificate error, and prints
the error. If the handshake succeeds, it ensures that the server sent back 
a certificate.

\begin{lstlisting}[style=code,language=C,numbers=none,caption={}]
// Do the SSL handshake
if(SSL_do_handshake(ssl) <= 0) {
	// SSL Handshake failed

	long verify_err = SSL_get_verify_result(ssl);
	if (verify_err != X509_V_OK) { 
		// It failed because the certificate chain validation failed
		fprintf(stderr, "Certificate chain validation failed: %s\n", X509_verify_cert_error_string(verify_err));
		return -1;
	}

// Recover the server's certificate
server_cert =  SSL_get_peer_certificate(ssl);
if (server_cert == NULL) {
	// The handshake was successful although the server did not provide a certificate
	// Most likely using an insecure anonymous cipher suite... get out!
	return -1;
}
\end{lstlisting}

If everything passes, the client knows a valid certificate has been received.
But who is the client talking to ?

%--------------------------------------------------
\section{Validating the server hostname}
\label{sec:hostname}
Once the certificate chain has been validated, the TLS client has to verify
the identity of the server by ensuring that the server certificate was signed
for the hostname to which it is trying to connect to. 

The identity of the server can be specified in the {\em Subject Alternative
Name} extension or in the {\em Common Name} within the {\em Subject} field of
the certificate. According to RFC
6125,\footnote{\url{https://tools.ietf.org/html/rfc6125}} if a certificate
contains the {\em Subject Alternative Name} extension, the {\em Common Name}
field should be ignored when validating the server's identity.


%--------------------------------------------------
\subsection{OpenSSL shortcomings}

The biggest challenge with hostname validation is that prior to version
1.1.0\footnote{This version has not been released yet.}, OpenSSL does not
provide any function to perform this critical task. Therefore, it is up to
application developers to properly extract relevant fields from the server's
certificate and compare them with the hostname the TLS client wants to connect
to, in order to validate the server's identity.

The remainder of this paper focuses on correctly implementing hostname
validation.
%--------------------------------------------------
\subsection{The {\em Subject Alternative Name} extension}

\subsubsection{Extracting hostnames from the {\em Subject Alternative Name} 
    extension}

If present in the certificate, the {\em Subject Alternative Name} extension
can contain various fields including DNS names, IP addresses, email addresses
or X.400 addresses. In the context of validating a TLS server's identity, DNS
names must be extracted from the extension and compared with the server's
expected hostname.


\subsubsection{Sample Code}

X.509 extensions are complex, and the OpenSSL functions to manipulate them are
poorly documented. As there are numerous ways to extract names from the {\em
Subject Alternative Name} extension, the following sample code tries to make
it as concise and simple as possible. It extracts DNS names from the
extension, and compares each of them with the expected hostname until either a
match is found or all the names within the extension have been tested.

\begin{lstlisting}[style=code,language=C,numbers=none,caption={}]
typedef enum {
	MatchFound,
	MatchNotFound,
	NoSANPresent,
	MalformedCertificate,
	Error
} HostnameValidationResult;

/**
* Tries to find a match for hostname in the certificate's Subject Alternative Name extension.
*
* Returns MatchFound if a match was found.
* Returns MatchNotFound if no matches were found.
* Returns MalformedCertificate if any of the hostnames had a NUL character embedded in it.
* Returns NoSANPresent if the SAN extension was not present in the certificate.
*/
static HostnameValidationResult matches_subject_alternative_name(const char *hostname, const X509 *server_cert) {
	HostnameValidationResult result = MatchNotFound;
	int i;
	int san_names_nb = -1;
	STACK_OF(GENERAL_NAME) *san_names = NULL;

	// Try to extract the names within the SAN extension from the certificate
	san_names = X509_get_ext_d2i((X509 *) server_cert, NID_subject_alt_name, NULL, NULL);
	if (san_names == NULL) {
		return NoSANPresent;
	}
	san_names_nb = sk_GENERAL_NAME_num(san_names);

	// Check each name within the extension
	for (i=0; i<san_names_nb; i++) {
		const GENERAL_NAME *current_name = sk_GENERAL_NAME_value(san_names, i);

		if (current_name->type == GEN_DNS) {
			// Current name is a DNS name, let's check it
			char *dns_name = (char *) ASN1_STRING_data(current_name->d.dNSName);

			// Make sure there isn't an embedded NUL character in the DNS name
			if (ASN1_STRING_length(current_name->d.dNSName) != strlen(dns_name)) {
				result = MalformedCertificate;
				break;
			}
			else { // Compare expected hostname with the DNS name
				if (strcasecmp(hostname, dns_name) == 0) {
					result = MatchFound;
					break;
				}
			}
		}
	}
	sk_GENERAL_NAME_pop_free(san_names, GENERAL_NAME_free);

	return result;
}
\end{lstlisting}


%--------------------------------------------------
\subsection{The {\em Common Name} field}

\subsubsection{Extracting the {\em Common Name} field}

If the {\em Subject Alternative Name} extension is not part of the
certificate, the {\em Common Name} within the {\em Subject} field must be
checked against the expected hostname.

\subsubsection{Sample code}

The following sample code extracts the {\em Common Name} from the server's
certificate and compares it with the expected hostname:

\begin{lstlisting}[style=code,language=C,numbers=none,caption={}]
/**
* Tries to find a match for hostname in the certificate's Common Name field.
*
* Returns MatchFound if a match was found.
* Returns MatchNotFound if no matches were found.
* Returns MalformedCertificate if the Common Name had a NUL character embedded in it.
* Returns Error if the Common Name could not be extracted.
*/
static HostnameValidationResult matches_common_name(const char *hostname, const X509 *server_cert) {
	int common_name_loc = -1;
	X509_NAME_ENTRY *common_name_entry = NULL;
	ASN1_STRING *common_name_asn1 = NULL;
	char *common_name_str = NULL;

	// Find the position of the CN field in the Subject field of the certificate
	common_name_loc = X509_NAME_get_index_by_NID(X509_get_subject_name((X509 *) server_cert), NID_commonName, -1);
	if (common_name_loc < 0) {
		return Error;
	}

	// Extract the CN field
	common_name_entry = X509_NAME_get_entry(X509_get_subject_name((X509 *) server_cert), common_name_loc);
	if (common_name_entry == NULL) {
		return Error;
	}

	// Convert the CN field to a C string
	common_name_asn1 = X509_NAME_ENTRY_get_data(common_name_entry);
	if (common_name_asn1 == NULL) {
		return Error;
	}			
	common_name_str = (char *) ASN1_STRING_data(common_name_asn1);

	// Make sure there isn't an embedded NUL character in the CN
	if (ASN1_STRING_length(common_name_asn1) != strlen(common_name_str)) {
		return MalformedCertificate;
	}

	// Compare expected hostname with the CN
	if (strcasecmp(hostname, common_name_str) == 0) {
		return MatchFound;
	}
	else {
		return MatchNotFound;
	}
}
\end{lstlisting}

\subsection{Wrapping up hostname validation}

The following sample code makes use of the previously defined functions to
perform hostname validation, given an expected hostname and the server's
certificate. If the validation is successful, the TLS client has authenticated
the server and the TLS connection is secure; the client can start sending 
application data.

\begin{lstlisting}[style=code,language=C,numbers=none,caption={}]
/**
* Validates the server's identity by looking for the expected hostname in the
* server's certificate. As described in RFC 6125, it first tries to find a match
* in the Subject Alternative Name extension. If the extension is not present in
* the certificate, it checks the Common Name instead.
*
* Returns MatchFound if a match was found.
* Returns MatchNotFound if no matches were found.
* Returns MalformedCertificate if any of the hostnames had a NUL character embedded in it.
* Returns Error if there was an error.
*/
HostnameValidationResult validate_hostname(const char *hostname, const X509 *server_cert) {
	HostnameValidationResult result;

	if((hostname == NULL) || (server_cert == NULL))
		return Error;

	// First try the Subject Alternative Names extension
	result = matches_subject_alternative_name(hostname, server_cert);
	if (result == NoSANPresent) {
		// Extension was not found: try the Common Name
		result = matches_common_name(hostname, server_cert);
	}

	return result;
}
\end{lstlisting}


%--------------------------------------------------
\section{Known limitations}
\label{sec:limitations}
The certificate validation guidelines described above are intended to work for
most applications. However, they have some limitations, described in the
following subsections.


\subsection{Revocation}

If a certificate is compromised, the CA that issued the certificate can
invalidate it before it expires using two mechanisms: Certificate Revocation
lists (CRL) and the Online Certificate Status Protocol (OCSP). During the TLS
handshake, clients can query the CA using those mechanisms to ensure that the
server certificate has not been revoked. The certificate validation guidelines
described in this paper do not check for revocation. 

Current revocation mechanisms suffer from various
problems\footnote{\url{http://www.imperialviolet.org/2011/03/18/revocation.html}} 
that are outside the scope of this paper. In the name of usability, those
issues have led browsers to do soft-fail checking, where the browser checks 
for revocation but does nothing if a problem occurs (for example if the
OCSP responder cannot be reached).

Still, not performing revocation checking is far from ideal and definitely
weakens the security of the TLS channel. This situation is also evolving, and
one possible solution to fix revocation is a mechanism called ``OCSP
Stapling''\footnote{\url{https://en.wikipedia.org/wiki/OCSP_stapling}}, which
was added to OpenSSL 0.9.8g. Guidelines on how to implement revocation
checking with OpenSSL using OCSP stapling may be described in a future version
of this paper.


\subsection{Wildcard certificates}

The hostname validation functions described in this paper only perform a
string comparison between the expected hostname and names extracted from the
certificate. RFC 2818\footnote{\url{https://www.ietf.org/rfc/rfc2818.txt}}
specifies that ``names may contain the wildcard character * which is considered
to match any single domain name component or component fragment''.

Supporting wildcard certificates requires manually parsing the name to find
the wildcard character, ensuring that it is in a valid location within the
domain, and then trying to match the pattern with the server's expected hostname.

Performing wildcard certificate validation is not trivial, and it is possible
to get an idea on how it should be done by looking at NSS'\footnote{\url{http://mxr.mozilla.org/mozilla/source/security/nss/lib/certdb/certdb.c}} 
or libCURL\footnote{\url{https://github.com/bagder/curl/blob/master/lib/ssluse.c}}'s implementations.

It should be noted that a domain administrator can sometimes avoid deploying a wildcard
certificate by putting all the possible names in the 
{\em Subject Alternative Name} extension, or using multiple certificates along 
with {\em Server Name Indication}\footnote{\url{https://en.wikipedia.org/wiki/Server_Name_Indication}}


\subsection{Other various limitations} 

Other scenarios where the validation code will not work include:

\begin{itemize}
    \item Certificates issued for IP addresses.
    \item More complex certificates, for example:
    \begin{itemize}
        \item Certificates that have multiple 
        {\em Common Names}.\footnote{As described in RFC 6125: 
            \url{https://tools.ietf.org/html/rfc6125}}
        \item Certificates with a "Name Constraints" 
            extension\footnote{The "Name Constraints" extension is not properly supported by OpenSSL prior to version 1.0.0.}.
    \end{itemize}
    \item Servers performing client authentication.
    \item Exotic cipher suites such as the SRP or PSK cipher suites.
\end{itemize}

Readers are encouraged to contact the author if any major functionality is
missing from this paper.

%--------------------------------------------------
\section{Conclusion}

Following the guidelines described in this paper will ensure that your TLS
client is securely connecting to its intended server, thereby preventing 
attackers on the network from performing man-in-the-middle attacks.

On the server side, Ivan Ristic's ``SSL/TLS Deployment Best Practices 
Guide''\footnote{\url{https://www.ssllabs.com/downloads/SSL_TLS_Deployment_Best_Practices_1.0.pdf}}
provides excellent information on how to deploy and configure a TLS server.

The full code for our test client including utility functions to validate the
server's hostname is available at:
\url{https://github.com/iSECPartners/ssl-conservatory/}.


\section{Acknowledgments}

Thanks to David Thiel, Alex Garbutt, Chris Palmer, Paul Youn, Andy Grant, Gabe
Pike and  Chris Palmer for reviewing various versions of this paper. Their
valuable  feedback is very much appreciated. Thanks to B.J Orvis for providing
an amazing SSL/TLS test suite with relevant test certificates.

\pagebreak
\appendix

% You can also include external files, handy if you're using something
% like epydoc, which can output LaTeX
%\include{appendix}

% I used footnotes instead of references
%\bibliographystyle{abbrv}
%\bibliography{whitepaper}

\end{document}
